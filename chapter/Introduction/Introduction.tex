%---------------------------------------------------------------------------------------------------
% Einf�hrung
%---------------------------------------------------------------------------------------------------
\newpage
%\part{Anfang}
\chapter{Introduction}
\label{cap:Ein}
This bachelorthesis describes the development process of a native Android application.
Today life without smart-phones has become unimaginable. It has infiltrated nearly every possible part of humans society. \textit{"In the year 2018 two out of three of the people worldwide and 81\% in Germany owns a smart-phone."} \cite{Schobelt17}
% https://www.wuv.de/digital/weltweite_smartphone_verbreitung_steigt_2018_auf_66_prozent

Applications look like a technical promise of the future. But reminiscing the past is very important to mankind. Like Elie Wiesel once said, \textit{"without memory, there is no culture. Without memory, there would be no civilization, no society, no future."} \cite{Wiesel}

The Civitas project will discover the history of urbanity. Getting knowledge from the past, to answer questions of the present. Civitas asks for the origin of urbanity, the organization of social life and the role of the cities. Nowadays, more than 50\% of humankind lives in cities. The goal is to allow interested people to travel through the Roman Empire and experience the Roman culture again. To connect the history with modern technologies, the Civitas application was invented. It should store historical monuments of the Roman Empire within the context of an Android app.

% Frau Panzram �ber CIVITAS
% https://www.geschichte.uni-hamburg.de/arbeitsbereiche/alte-geschichte/personen/panzram.html

\section{Initial Situation}
An existing Civitas application for historically interested people was delivered at the beginning of this study. This application came along with limited possibilities in several respects. 
If and how such conflicts can be resolved is the subject of this thesis.


\section{Criteria For Success}

According to the title of this bachelorthesis, the existing Civitas app should get an extension. Consulting the previous developer, and studying the former thesis gave an overview of missing features and their degree of prioritization. These features are a topic of this study and will be discussed in detail later. Beside missing features, it turns out, that especially the back-end structure was a remarkable and time-consuming obstacle, for continuing the Civitas app. Avoiding a repetition of this time consumption again, in this, and the following development processes, an easy entrance is a criterion as well. 
