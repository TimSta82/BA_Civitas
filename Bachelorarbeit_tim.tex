

%---------------------------------------------------------------------------------------------------
% Voreinstellungen (Layout, neue Befehle, etc.)
%---------------------------------------------------------------------------------------------------

\input{einstellungen/grundeinstellungen}															% Die stilistischen Parameter

%--------------------------------------------------------------------------------------------------- 
% Anfang des Schriftst�cks
%---------------------------------------------------------------------------------------------------	
\begin{document}

%--------------------------------------------------------------------------------------------------- 
% Erstellen des Deck- und des Titelblatts
%---------------------------------------------------------------------------------------------------
		
	\createCoverAndTitlePage{Bachelor}																			% Art der Arbeit
													{thesis}																			% Bezeichnung arbeit oder thesis 
													{Tim Staats}														% Author
													{Extending an Android App for a Multi-user Geodatabase of Historical Monuments}											  % Titel				
													{Informations- und Elektrotechnik}						% Studiengang
													{Prof.\ Dr.\ rer. \ nat. Henning Dierks}					% Erstgutachter
													{Prof.\ Dr.\ Marc Hensel}
			% Zweitgutachter
													
% - -------------------------------------------------------------------------
  \createAbstract					{Bachelor}												%							% Art der Arbeit
													{thesis}								%											% Bezeichnung arbeit oder thesis 
													{Tim Staats}					%							% Author
													{Weiterentwicklung einer Geodatenbasierten Multi-Nutzer Android App f�r historische Monumente}											  % Titel				
													{Extending an Android App for a Multi-user Geodatabase of Historical Monuments}							
													% Titel Englisch
													{Java, Android, Webapplication, Backendless, Geodaten, Nutzerservice, historische Monumente, Google Map, BaaS}																	% Stichworte
													{Java, Android, Webapplication, Backendless, Geodata, Userservice, historical monuments, google map, Baas}																			% Keywords (Stichworte Emglisch)
													{Diese Arbeit dokumentiert die Weiterentwicklung einer Android Applikation. Die App bietet registrierten Nutzern die M�glichkeit historische Monumente zu erfassen und zu katalogisieren. Des  Weiteren visualisiert die App die katalogisierten Daten auf eine google Map und in Form von detailierten Listen. Im dazugeh�rigen Backend werden die Daten verwaltet und in einer Web Applikation visualisiert. Die daf�r erforderlichen Ma�nahmen, Konzepte, Probleme und deren L�sungen werden im folgenden beschrieben.
}																					% Kurzzusammenfassung
													{This thesis documents the development of an Android Application. The app offers the opportunity to registered users to capture and catalog historical Monuments. Furthermore the app displays the data within a google map and in form of a detailed list. The corresponding Backend manages the data and also display it within a webapplication. The necessary measures, concepts, problems and their solutions are described below.}																			% Abstract (Kurzzusammenfassung Englisch)

												
%--------------------------------------------------------------------------------------------------- 
% Zusammenfassung
%---------------------------------------------------------------------------------------------------			
  									  													


%--------------------------------------------------------------------------------------------------- 
% Danksagung  
%---------------------------------------------------------------------------------------------------	
	%\input{standard/danke}  																							

%--------------------------------------------------------------------------------------------------- 
% Verzeichnisse
%---------------------------------------------------------------------------------------------------	
  \tableofcontents                              												% Inhaltsverzeichnis
	%\listoftables                                 												% Tabellenverzeichnis
	%\listoffigures                                												% Abbildungsverzeichnis  
	
%--------------------------------------------------------------------------------------------------- 
% Der erste Teil der Arbeit:
%---------------------------------------------------------------------------------------------------
	%---------------------------------------------------------------------------------------------------
% Einf�hrung
%---------------------------------------------------------------------------------------------------
\newpage
%\part{Anfang}
\chapter{Introduction}
\label{cap:Ein}
This bachelorthesis describes the development process of a native Android application.
Today life without smart-phones has become unimaginable. It has infiltrated nearly every possible part of humans society. \textit{"In the year 2018 two out of three of the people worldwide and 81\% in Germany owns a smart-phone."} \cite{Schobelt17}
% https://www.wuv.de/digital/weltweite_smartphone_verbreitung_steigt_2018_auf_66_prozent

Applications look like a technical promise of the future. But reminiscing the past is very important to mankind. Like Elie Wiesel once said, \textit{"without memory, there is no culture. Without memory, there would be no civilization, no society, no future."} \cite{Wiesel}

The Civitas project will discover the history of urbanity. Getting knowledge from the past, to answer questions of the present. Civitas asks for the origin of urbanity, the organization of social life and the role of the cities. Nowadays, more than 50\% of humankind lives in cities. The goal is to allow interested people to travel through the Roman Empire and experience the Roman culture again. To connect the history with modern technologies, the Civitas application was invented. It should store historical monuments of the Roman Empire within the context of an Android app.

% Frau Panzram �ber CIVITAS
% https://www.geschichte.uni-hamburg.de/arbeitsbereiche/alte-geschichte/personen/panzram.html

\section{Initial Situation}
An existing Civitas application for historically interested people was delivered at the beginning of this study. This application came along with limited possibilities in several respects. 
If and how such conflicts can be resolved is the subject of this thesis.


\section{Criteria For Success}

According to the title of this bachelorthesis, the existing Civitas app should get an extension. Consulting the previous developer, and studying the former thesis gave an overview of missing features and their degree of prioritization. These features are a topic of this study and will be discussed in detail later. Beside missing features, it turns out, that especially the back-end structure was a remarkable and time-consuming obstacle, for continuing the Civitas app. Avoiding a repetition of this time consumption again, in this, and the following development processes, an easy entrance is a criterion as well. 

	
%---------------------------------------------------------------------------------------------------	
% Der zweite Teil der Arbeit:
%---------------------------------------------------------------------------------------------------
	%---------------------------------------------------------------------------------------------------
% Analysis
%---------------------------------------------------------------------------------------------------
\newpage
%\part{Anfang}
\chapter{Analysis}\label{cap:Analysis}
ich analysiere!

%---------------------------------------------------------------------------------------------------
% Der dritte Teil der Arbeit
%---------------------------------------------------------------------------------------------------
	%---------------------------------------------------------------------------------------------------
% Design
%---------------------------------------------------------------------------------------------------
\newpage
%\part{Anfang}
\chapter{Design}\label{cap:Design}
ich designe!
														
%---------------------------------------------------------------------------------------------------	
% Der vierter Teil der Arbeit
%---------------------------------------------------------------------------------------------------
	%---------------------------------------------------------------------------------------------------
% Realization
%---------------------------------------------------------------------------------------------------
\newpage
%\part{Anfang}
\chapter{Realization}\label{cap:Realization}
\subsection{Application Class}
An Application Class is called before the MainActivity is called for the first time. Due to the usage of Backendless (BaaS) the applicationId, Api Key and the serverUrl are stored in the ApplicationClass and connects the Webapplication with the Android Application. It also contains variables for the user credentials saved in the Backendless.user table and the Artefact table. 

\subsection{Backendless}
Backendless provides the server data to the app when it is requested. This happens in case of registration, login, artefact creation, file upload, file download, geolocating marker, edit artefact, etc.
Independently which data is needed, an asynchronous Backendless request requires the implemention of two interface methods. The asyncCallback methods "handleResponse()" and "handleFault()" are recommended to keep the main thread free and increasing the application performance. "handleFault()" is called if something went wrong. "handleResponse()" is called if the request was successful and provides the requested data.

\subsubsection{User table}
With Backendless one can create data tables with ease. The user table should contain certain properties such as name, email and password. To get a table like that one calls the Backendless.UserService.register() and pass a BackendlessUser object as argument.

An exemplary interaction between app and backendless is displayed here in case of user registration.

\fbox{
\lstinputlisting[label={code:backendless_register} ,caption={Backendless user registration},captionpos=b, language = java,  numbers = left]{program/backendless_register.java}
}

The method handleResponse() returns to the LoginFragment, waiting for the user to login. From now on the user table is listed within the Backendless console (Webapplication) \ref{fig:backendlessConsoleUserTable}.


\subsubsection{LoginFragment}
This Fragment is the first view which appears to the user. It provides three possibilities. 
\begin{itemize}
\item Login
\item Register
\item Password recovery
\end{itemize}
The first step for a new user is moving to the RegisterFragment via button click.
If the registration is accomplished, one can login or reset password.
Successful login leads to the Map, which is the main part of the application. It keeps the user logged in until the logout button is clicked, hence the next time Civitas is started it navigates directly to the Map.

The Login progress is displayed here. 

\fbox{
\lstinputlisting[label={code:backendless_login} ,caption={Backendless user login},captionpos=b, language = java,  numbers = left]{program/backendless_login.java}
}

A basic Backendless request implements two methods, handleResponse() and handleFault(). If request is successful, the response contains the user data. This data is stored in the Applicationclass.user for further purpose.

% --------------------------------------------------------------------------------------------------
% webapplication
% --------------------------------------------------------------------------------------------------

\section{Web Application - Backendless Console}

\subsection{User's}
email templates
\subsection{Data}
\subsubsection{User table}
The user table contains the same properties like the BackendlessUser object \ref{code:backendless_register} where it is created from plus the properties \textit{created} and \textit{updated} which where provided automatically from the Backendless.

\begin{figure}[H]
	\centering \includegraphics[width=0.8\textwidth]{backendless_console_user_table.png}
	\caption[backendlessConsoleUserTable]{Backendless console user table}
	\label{fig:backendlessConsoleUserTable}
\end{figure}
\footnotetext{URL: https://www.banksy.org [cited 22 August 2018]}

\subsubsection{Artefact table}
contains app Table
contains system data
\subsection{Files}
images and audios
\subsection{Geolocation}
displays artefact coordinates

ich realisiere!
														
%---------------------------------------------------------------------------------------------------	
% Der fuenfter Teil der Arbeit
%---------------------------------------------------------------------------------------------------
	%---------------------------------------------------------------------------------------------------
% Tests
%---------------------------------------------------------------------------------------------------
\newpage
%\part{Anfang}
\chapter{Tests}\label{cap:Tests}

\section{JUnit}
\begin{itemize}
\item test ApplicationClass.mArtefactList
\item test CategoryList 
\end{itemize}

\section{Integration Test}
buttons work well
ich Teste
														
%---------------------------------------------------------------------------------------------------	
% Der sechster Teil der Arbeit
%---------------------------------------------------------------------------------------------------
	%---------------------------------------------------------------------------------------------------
% Conclusion
%---------------------------------------------------------------------------------------------------
\newpage
%\part{Anfang}
\chapter{Conclusion}\label{cap:Conclusion}
ich fasse zusammen
\section{Furthermore}
On the one hand there is a working app with a lot of functions such as. 
\begin{itemize}
\item Navigation through the app
\item Screen returns to last focus
\item Webapplication for administrational purpose (admin control panel)
\item Password recovery
\item List overview
\item Complex filter system within list and map
\item Create artefacts (everybody)
\item Edit artefacts (authorized users)
\item Delete artefacts (authorized users)
\end{itemize}
But on the flip side of things, there is still a lot of work to do.
\begin{itemize}
\item MP3
\item Share
\item Rating system
\item Thumbnail within marker info window
\item Edit user credentials
\item Multiple image related to a single artefact
\item Comment function for artefacts
\item Retrieve artefacts in relation to the device location

\end{itemize}
														
%---------------------------------------------------------------------------------------------------	
% Literaturverzeichnis
%---------------------------------------------------------------------------------------------------		
  \bibliographystyle{IEEEtranN}   % was dinat        		    														% Anpassung an deutsche Zitierweise
                                          														% Alphabetische Sortierung, Abk�rzungen
  \bibliography{literatur/literatur}    %KAY: WAR DRINNEN   														% Literaturverzeichnis
%  \input{literatur/literatur}			 %KAY: WAR DRINNEN   																		% hier k�nnen alle Schriftst�cke aufgef�hrt werden, die nicht zitiert, aber dennoch nennenswert sind!
  

											

%---------------------------------------------------------------------------------------------------	
% Anh�nge
%---------------------------------------------------------------------------------------------------	
	\appendix  							%KAY: WAR DRINNEN   
	  	\input{chapter/Endpart/Appendix.tex}
% \input{anhang/hilfsmittel/hilfsmittel}															% Anhang A: Hilfsmittel zur Erstellung
																																			% 					dieser Arbeit
%	\input{anhang/quellcode/quellcode}																	% Anhang B: Quellcode

%---------------------------------------------------------------------------------------------------	
% Glossar
%---------------------------------------------------------------------------------------------------	
	\printnomenclature

%---------------------------------------------------------------------------------------------------	
% Stichwortverzeichnis
%---------------------------------------------------------------------------------------------------	
	\printindex
	
%---------------------------------------------------------------------------------------------------	
% Erkl�rung �ber Selbstst�ndigkeit
%---------------------------------------------------------------------------------------------------		
	\asurency	

%--------------------------------------------------------------------------------------------------- 
% Ende des Schriftst�cks
%--------------------------------------------------------------------------------------------------- 
\end{document}
